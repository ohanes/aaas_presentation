\documentclass{beamer}
   \usetheme{PaloAlto}
   % make math fonts look like article
   \usefonttheme[onlymath]{serif} 

   % Title page metadata
   \title[Neuro Fuzzy Hybridization]
   {A Neuro Fuzzy Hybridization Model for Nonlinear Control of a Fixed Wing UAV}
   \subtitle{Human-like Reasoning within Connectionist Structure}
   \author[Dadian]
   {Ohanes ~Dadian\inst{1} \\
   {Software Engineer} \\
   {Northrop Grumman Corporation}}
   \subject{Computer Science}

\begin{document}
   \frame{\titlepage}

   \section{Topics}

   \begin{frame}
      \frametitle{What is Fuzzy Logic?}
      \begin{itemize}
         \item Fuzzy logic is a form of many-valued logic.
         \begin{itemize}
            \item Truth tables of variables may be any real number between 0 and 1.
            \item Provides capability of deriving partial truth.
            \item Fuzzy logic was introduced in 1965 by Lofti Zadeh.
         \end{itemize}
         \item Came out of early research in fuzzy set theory.
         \item Dating back to the 1920s with Lukasiewicz and Tarski.
         \item Popular applications in control theory and artificial intelligence.
      \end{itemize}
   \end{frame}

   \begin{frame}
      \frametitle{Fuzzification}
      \begin{itemize}
         \item The following steps allow data to be fuzzified.
         \begin{itemize}
            \item Fuzzify all input values into fuzzy membership functions.
            \item Execute all applicable rules into the rulebase to compute fuzzy output functions.
            \item De-fuzzify the fuzzy output functions to get "crisp" output values.
         \end{itemize}
      \end{itemize}
   \end{frame}

   \begin{frame}
      \frametitle{Fuzzy Logic Example}
      \begin{columns}[T]
         \begin{column}[T]{5cm}
            \begin{itemize}
               \item Let us run through an example of this process against the concept of temperature.
               \item Per the image, we have three expressions: cold, warm, and hot.
               \item These expressions map to functions mapping to the temperature scales.
            \end{itemize}
         \end{column}
         \begin{column}[T]{5cm}
            \includegraphics[height=2cm]{fuzzy_logic_temperature}
         \end{column}
      \end{columns}
   \end{frame}

   \begin{frame}
      \frametitle{Fuzzy Logic Example}
      \begin{itemize}
         \item A point on this scale has three truth values.
         \begin{itemize}
            \item One for each of the three functions.
         \end{itemize}
         \item The vertical arrow depicts a particular temperature.
         \item Red arrow points to zero, i.e. not hot.
         \item Orange arrow points at 0.2, i.e. slightly warm.
         \item Blue arrow points at 0.8, i.e. fairly cold.
      \end{itemize} 
   \end{frame}

   \begin{frame}
      \frametitle{Fuzzy Sets}
      \begin{columns}[T]
         \begin{column}[T]{5cm}
            \begin{itemize}
               \item Fuzzy sets are generally defined as traingle or trapezoid-shaped curves.
               \begin{itemize}
                  \item Each value has a slope where value is increasing, peak of 1, and a slope where the value is decreasing.
                  \item Can also be modeled as a sigmoid.
               \end{itemize}
            \end{itemize}
         \end{column}
         \begin{column}[T]{5cm}
            \begin{center}
               \underline{Logistic Sigmoid} \\
               $ x = \frac{e^{x}}{e^{x}+1} $ \\
               $ S(x) + S(-x) =  1 $ \\
               $ (S(x) + S(-x)) \times (S(y) + S(-y)) \times (S(z) + S(-z)) = 1 $ 
            \end{center}
         \end{column}
      \end{columns}
   \end{frame}
   
   \begin{frame}
      \frametitle{Fuzzy Logic Operators}
      \begin{columns}[T]
        \begin{column}[T]{5cm}
            \begin{itemize}
               \item Fuzzy logic works with membership values in a way that mimics boolean logic.
               \item There are altertives for the operator counterparts.
               \item There are other operators available called hedges.
               \item These are generally adverbs such as very, or somewhat, which modify the meaning of a set using a mathematical formula.
            \end{itemize}
        \end{column}
        \begin{column}[T]{5cm}
            \begin{tabular}{| c | c |}
               \hline
               Boolean & Fuzzy \\
               \hline
               AND(x,y) & MIN(x,y) \\
               OR(x,y)  & MAX(x,y) \\
               NOT(x) & 1 - x \\
               \hline
            \end{tabular}
        \end{column}
      \end{columns}
   \end{frame}

   \begin{frame}
      \begin{columns}[T]
         \begin{column}[T]{5cm}
            \begin{itemize}
               \item A fuzzy logic function represents a disjunction of constituents of minimum, where a constituent of minimum is a conjunction of variables of the current area greater than or equal to the function value in this area.
               \item Another set of AND/OR operators are based on multiplication.
               \item 1-(1-x)*(1-y) comes from this.
            \end{itemize}
         \end{column}
         \begin{column}[T]{5cm}
            x AND y = x*y \\
            x OR y = 1-(1-x)*(1-y) = x+y-x*y \\
            x OR y = NOT( AND( NOT(x), NOT(y) ) ) \\
            x OR y = NOT( AND(1-x, 1-y) ) \\
            x OR y = NOT( (1-x)*(1-y) ) \\
            x OR y = 1-(1-x)*(1-y)
         \end{column}
      \end{columns}
   \end{frame}

   \begin{frame}
      \frametitle{IF-THEN Rule Sets}
      \begin{columns}[T]
         \begin{column}[T]{5cm}
            \begin{itemize}
               \item IF-THEN rules map input or computed truth values to desired output truth values.
               \item Given a certain temperature, the fuzzy variable hot has a certain truth value, which is copied to the high variable.
               \item If an output variable exists in many THEN parts, then values from the IF parts are combined using the OR operator.
            \end{itemize}
         \end{column}
         \begin{column}[T]{5cm}
            IF temperature IS very cold THEN fanSpeed is stopped \\
            IF temperature IS cold THEN fanSpeed is slow \\
            IF temperature IS warm THEN fanSpeed is moderate \\
            IF temperature IS hot THEN fanSpeed is high
         \end{column}
     \end{columns}
   \end{frame}

   \begin{frame}
      \frametitle{Defuzzification}
      \begin{itemize}
         \item The goal is to get a continuous variable from fuzzy truth values.
         \item For each truth value, cut the membership function at this value
         \item Combine the resulting curves using the OR operator
         \item Find the center-of-weight of the area under the curve
         \item The x position of this center is then the final output.
      \end{itemize}
   \end{frame}
\end{document}
